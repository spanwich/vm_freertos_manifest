\documentclass{article}
\usepackage[utf8]{inputenc}
\usepackage{geometry}
\usepackage{listings}
\usepackage{xcolor}
\usepackage{hyperref}

\geometry{margin=1in}

% Code formatting
\lstset{
    basicstyle=\ttfamily\small,
    breaklines=true,
    frame=single,
    backgroundcolor=\color{gray!10},
    numbers=left,
    numberstyle=\tiny,
    showstringspaces=false,
    keywordstyle=\color{blue},
    commentstyle=\color{green!50!black},
    stringstyle=\color{red}
}

\title{Stack 2 Setup Guide: ARM32 Linux Hypervisor for FreeRTOS Virtualization}
\author{PhD Research: Hypervisor Comparison Framework}
\date{\today}

\begin{document}

\maketitle

\section{Overview}

This document provides step-by-step instructions for setting up Stack 2 in the hypervisor comparison research framework. Stack 2 implements the traditional Linux hypervisor approach for running FreeRTOS as a virtualized guest.

\subsection{Research Architecture}

\begin{itemize}
    \item \textbf{Stack 1}: QEMU $\rightarrow$ seL4 $\rightarrow$ FreeRTOS (formally verified microkernel)
    \item \textbf{Stack 2}: QEMU $\rightarrow$ Linux ARM32 $\rightarrow$ FreeRTOS (traditional hypervisor)
\end{itemize}

\subsection{Stack 2 Components}

\begin{itemize}
    \item \textbf{Host}: Development machine
    \item \textbf{L1 Hypervisor}: QEMU system emulation
    \item \textbf{Guest OS}: Debian ARM32 Linux (acts as L2 hypervisor)
    \item \textbf{Nested Guest}: FreeRTOS RTOS
    \item \textbf{Architecture}: ARM32 (cortex-a15) for consistency
\end{itemize}

\section{Prerequisites}

\begin{itemize}
    \item QEMU system emulation for ARM32
    \item FreeRTOS binary: \texttt{freertos\_image.bin}
    \item Debian ARM32 image: \texttt{dqib\_armhf-virt}
    \item SSH client for system access
\end{itemize}

\section{Step-by-Step Setup}

\subsection{Step 1: Navigate to ARM32 Linux Directory}

\begin{lstlisting}[language=bash]
cd /home/konton-otome/phd/linux_vm_comparison/minimal_rootfs/rootfs/dqib_armhf-virt
\end{lstlisting}

\subsection{Step 2: Boot ARM32 Linux System}

Execute the following QEMU command to start the ARM32 Linux system:

\begin{lstlisting}[language=bash]
qemu-system-arm \
  -machine virt \
  -cpu cortex-a15 \
  -m 2047M \
  -device virtio-blk-device,drive=hd \
  -drive file=image.qcow2,if=none,id=hd \
  -device virtio-net-device,netdev=net \
  -netdev user,id=net,hostfwd=tcp::2223-:22 \
  -kernel kernel \
  -initrd initrd \
  -nographic \
  -append "root=LABEL=rootfs console=ttyAMA0"
\end{lstlisting}

\textbf{Note}: SSH port 2223 is used to avoid conflicts with other QEMU instances.

\subsection{Step 3: System Login Options}

\subsubsection{Console Login}
\begin{itemize}
    \item Username: \texttt{root}, Password: \texttt{root}
    \item Username: \texttt{debian}, Password: \texttt{debian}
\end{itemize}

\subsubsection{SSH Login (Recommended)}
\begin{lstlisting}[language=bash]
ssh -p 2223 root@localhost
\end{lstlisting}

SSH provides better terminal emulation than the QEMU console.

\subsection{Step 4: Transfer FreeRTOS Binary}

After the ARM32 Linux system is running and SSH is accessible:

\begin{lstlisting}[language=bash]
scp -P 2223 /home/konton-otome/phd/linux_kvm_setup/freertos_image.bin root@localhost:/home/
\end{lstlisting}

\section{Alternative File Transfer Methods}

\subsection{Method 1: Mount qcow2 Image (Pre-boot)}

\begin{lstlisting}[language=bash]
# Create mount point
sudo mkdir -p /mnt/arm32-rootfs

# Mount the qcow2 image
sudo modprobe nbd max_part=8
sudo qemu-nbd --connect=/dev/nbd0 image.qcow2
sudo mount /dev/nbd0p1 /mnt/arm32-rootfs

# Copy FreeRTOS binary
sudo cp /home/konton-otome/phd/linux_kvm_setup/freertos_image.bin \
       /mnt/arm32-rootfs/home/

# Unmount
sudo umount /mnt/arm32-rootfs
sudo qemu-nbd --disconnect /dev/nbd0
\end{lstlisting}

\subsection{Method 2: Additional Virtual Drive}

Add the following options to the QEMU command:

\begin{lstlisting}[language=bash]
-drive file=fat:rw:/home/konton-otome/phd/linux_kvm_setup,format=raw,id=usbdrive \
-device usb-storage,drive=usbdrive
\end{lstlisting}

\section{System Control}

\subsection{Exit QEMU}
Press \texttt{Ctrl+A} then \texttt{X} to exit QEMU.

\subsection{System Shutdown}
From within the ARM32 Linux system:
\begin{lstlisting}[language=bash]
shutdown -h now
\end{lstlisting}

\section{Next Steps}

Once the ARM32 Linux system is running with the FreeRTOS binary available:

\begin{enumerate}
    \item Install QEMU within the ARM32 Linux system
    \item Configure nested virtualization for FreeRTOS
    \item Run FreeRTOS as a nested guest
    \item Collect performance and behavior data
    \item Compare with Stack 1 (seL4) results
\end{enumerate}

\section{Architecture Verification}

\subsection{Consistency Check}
\begin{itemize}
    \item \textbf{FreeRTOS Binary}: ARM32 (cortex-a15 compatible)
    \item \textbf{Linux Host}: ARM32 (cortex-a15)
    \item \textbf{Nested Guest}: ARM32 (cortex-a15)
\end{itemize}

This ensures architectural consistency for meaningful hypervisor comparison research.

\subsection{Research Benefits}
\begin{itemize}
    \item \textbf{Same Guest Binary}: Identical FreeRTOS across both stacks
    \item \textbf{Same Host Architecture}: ARM32 cortex-a15 consistency
    \item \textbf{Controlled Environment}: Isolated comparison conditions
    \item \textbf{True Hypervisor Comparison}: seL4 vs Linux kernel approaches
\end{itemize}

\section{Troubleshooting}

\subsection{Common Issues}
\begin{itemize}
    \item \textbf{Port Conflicts}: Ensure port 2223 is not in use
    \item \textbf{SSH Keys}: Use password authentication initially
    \item \textbf{Disk Space}: Monitor available space in ARM32 system
    \item \textbf{Architecture Mismatch}: Verify ARM32 consistency
\end{itemize}

\subsection{Verification Commands}
\begin{lstlisting}[language=bash]
# Check architecture
uname -m
file /home/freertos_image.bin

# Check available space
df -h

# Verify QEMU installation
which qemu-system-arm
\end{lstlisting}

\end{document}