\documentclass[11pt,a4paper]{article}
\usepackage[utf8]{inputenc}
\usepackage[margin=1in]{geometry}
\usepackage{amsmath}
\usepackage{amsfonts}
\usepackage{amssymb}
\usepackage{listings}
\usepackage{xcolor}
\usepackage{hyperref}
\usepackage{fancyhdr}

% Code listing configuration
\lstset{
    basicstyle=\ttfamily\small,
    commentstyle=\color{gray},
    keywordstyle=\color{blue},
    stringstyle=\color{red},
    showstringspaces=false,
    breaklines=true,
    frame=single,
    numbers=left,
    numberstyle=\tiny\color{gray},
    backgroundcolor=\color{gray!10}
}

\pagestyle{fancy}
\fancyhf{}
\rhead{vm\_freertos Git Setup Guide}
\lfoot{PhD Research Documentation}
\rfoot{\thepage}

\title{Setting Up vm\_freertos as a Git Subproject: \\
Multiple Strategies for Research Code Management}

\author{PhD Research Documentation}
\date{\today}

\begin{document}

\maketitle

\begin{abstract}
This document provides comprehensive guidance for setting up your custom vm\_freertos implementation as a git-managed subproject within the CAmkES VM examples framework. Multiple strategies are presented, including git submodules, subtrees, and independent repositories, with recommendations based on different research workflow requirements.
\end{abstract}

\section{Overview of Strategies}

\subsection{Strategy Comparison}

\begin{table}[h]
\centering
\begin{tabular}{|l|p{3cm}|p{3cm}|p{3cm}|}
\hline
\textbf{Strategy} & \textbf{Pros} & \textbf{Cons} & \textbf{Best For} \\
\hline
Git Submodule & 
\begin{itemize}
\item Independent versioning
\item Clean separation
\item Easy sharing
\end{itemize} & 
\begin{itemize}
\item Complex workflows
\item Requires discipline
\item Learning curve
\end{itemize} & 
\begin{itemize}
\item Collaborative research
\item Publication code
\item Long-term projects
\end{itemize} \\
\hline
Git Subtree & 
\begin{itemize}
\item Simpler workflows
\item Single clone
\item No .gitmodules
\end{itemize} & 
\begin{itemize}
\item Merge complexity
\item History pollution
\item Less flexible
\end{itemize} & 
\begin{itemize}
\item Personal projects
\item Rapid development
\item Simple integration
\end{itemize} \\
\hline
Independent Repo & 
\begin{itemize}
\item Full independence
\item Custom workflows
\item Easy backup
\end{itemize} & 
\begin{itemize}
\item Manual sync
\item No integration
\item Isolation overhead
\end{itemize} & 
\begin{itemize}
\item Experimental work
\item Backup strategy
\item Portfolio projects
\end{itemize} \\
\hline
\end{tabular}
\caption{Git Strategy Comparison}
\end{table}

\section{Option 1: Git Submodule (Recommended for Research)}

Git submodules allow you to keep your vm\_freertos as a separate repository while linking it into the main project.

\subsection{Setup Process}

\begin{lstlisting}[caption=Setting Up Git Submodule]
# 1. Create and push your vm_freertos repository
cd /path/to/your/vm_freertos
git init
git add -A
git commit -m "Initial commit: FreeRTOS VM implementation"

# Push to your remote repository (GitHub, GitLab, etc.)
git remote add origin <your-remote-url>
git branch -M main  # Optional: use 'main' instead of 'master'
git push -u origin main

# 2. Remove local directory and add as submodule
cd /home/konton-otome/phd/camkes-vm-examples
rm -rf projects/vm-examples/apps/Arm/vm_freertos

# 3. Add as submodule
git submodule add <your-remote-url> projects/vm-examples/apps/Arm/vm_freertos

# 4. Commit the submodule configuration
git add .gitmodules projects/vm-examples/apps/Arm/vm_freertos
git commit -m "Add vm_freertos as submodule"
\end{lstlisting}

\subsection{Daily Workflow}

\begin{lstlisting}[caption=Working with Submodules]
# Updating your vm_freertos code
cd projects/vm-examples/apps/Arm/vm_freertos
# Make your changes
git add -A
git commit -m "Your changes"
git push

# Update parent repository to point to new commit
cd /home/konton-otome/phd/camkes-vm-examples
git add projects/vm-examples/apps/Arm/vm_freertos
git commit -m "Update vm_freertos submodule"

# Cloning project with submodules (for collaborators)
git clone --recursive <main-repo-url>
# OR
git clone <main-repo-url>
git submodule update --init --recursive
\end{lstlisting}

\subsection{Benefits for Research}

\begin{itemize}
\item \textbf{Independent Versioning}: Your research code has its own git history
\item \textbf{Easy Sharing}: Others can use just your vm\_freertos without the full CAmkES setup
\item \textbf{Publication Ready}: Clean, standalone code for paper submissions
\item \textbf{Backup Strategy}: Your code is safely stored in its own repository
\item \textbf{Branching}: Easy to experiment with different approaches
\end{itemize}

\section{Option 2: Git Subtree (Simpler Workflow)}

Git subtrees merge your subproject into the main repository while maintaining the ability to push/pull from the original.

\subsection{Setup Process}

\begin{lstlisting}[caption=Setting Up Git Subtree]
# 1. Create your remote vm_freertos repository first
# (Same as step 1 in submodule setup)

# 2. Remove local directory
cd /home/konton-otome/phd/camkes-vm-examples
rm -rf projects/vm-examples/apps/Arm/vm_freertos

# 3. Add as subtree
git subtree add --prefix=projects/vm-examples/apps/Arm/vm_freertos \
    <your-remote-url> main --squash

# The --squash option keeps history clean
\end{lstlisting}

\subsection{Daily Workflow}

\begin{lstlisting}[caption=Working with Subtrees]
# Make changes in vm_freertos directory
cd projects/vm-examples/apps/Arm/vm_freertos
# Edit files...

# Commit to main repository
cd /home/konton-otome/phd/camkes-vm-examples
git add projects/vm-examples/apps/Arm/vm_freertos
git commit -m "Update vm_freertos implementation"

# Push changes to vm_freertos repository
git subtree push --prefix=projects/vm-examples/apps/Arm/vm_freertos \
    <your-remote-url> main

# Pull updates from vm_freertos repository
git subtree pull --prefix=projects/vm-examples/apps/Arm/vm_freertos \
    <your-remote-url> main --squash
\end{lstlisting}

\section{Option 3: Independent Repository with Symlinks}

Keep vm\_freertos as a completely separate repository and link it into the CAmkES project.

\subsection{Setup Process}

\begin{lstlisting}[caption=Independent Repository Setup]
# 1. Move vm_freertos to separate location
mv /home/konton-otome/phd/camkes-vm-examples/projects/vm-examples/apps/Arm/vm_freertos \
   /home/konton-otome/phd/vm_freertos

# 2. Initialize git repository
cd /home/konton-otome/phd/vm_freertos
git init
git add -A
git commit -m "Initial commit: FreeRTOS VM implementation"
git remote add origin <your-remote-url>
git push -u origin main

# 3. Create symlink in CAmkES project
cd /home/konton-otome/phd/camkes-vm-examples/projects/vm-examples/apps/Arm
ln -s /home/konton-otome/phd/vm_freertos vm_freertos

# 4. Add symlink to gitignore in main project
echo "projects/vm-examples/apps/Arm/vm_freertos" >> .gitignore
\end{lstlisting}

\subsection{Benefits}

\begin{itemize}
\item \textbf{Complete Independence}: No git integration complexity
\item \textbf{Flexible Structure}: Organize your research code however you want
\item \textbf{Easy Backup}: Simple to backup separately
\item \textbf{Portfolio Building}: Standalone project for your GitHub profile
\end{itemize}

\section{Option 4: Repo-Managed Subproject (Advanced)}

Integrate your vm\_freertos into the repo manifest system used by CAmkES.

\subsection{Create Custom Manifest}

\begin{lstlisting}[language=xml, caption=Custom Manifest (custom-manifest.xml)]
<?xml version="1.0" encoding="UTF-8"?>
<manifest>
  <!-- Include the default CAmkES VM manifest -->
  <include name="default.xml" />
  
  <!-- Add your custom vm_freertos project -->
  <project name="vm_freertos" 
           path="projects/vm-examples/apps/Arm/vm_freertos" 
           remote="your-git-server"
           revision="main" />
</manifest>
\end{lstlisting}

\begin{lstlisting}[caption=Using Custom Manifest]
# Initialize with your custom manifest
repo init -u <your-manifest-repo> -m custom-manifest.xml
repo sync

# Now vm_freertos is managed by repo tool
\end{lstlisting}

\section{Recommendations by Use Case}

\subsection{For Active PhD Research (Recommended)}

\textbf{Use Git Submodule} because:
\begin{itemize}
\item Your research code remains independent and publishable
\item Easy to share specific versions with collaborators
\item Clean separation between your work and the framework
\item Supports branching for different experiments
\item Industry-standard approach for research codebases
\end{itemize}

\subsection{For Rapid Prototyping}

\textbf{Use Independent Repository with Symlinks} because:
\begin{itemize}
\item Fastest to set up and modify
\item No git complexity to learn
\item Easy to reorganize as project evolves
\item Simple backup and sharing
\end{itemize}

\subsection{For Collaborative Projects}

\textbf{Use Git Subtree} because:
\begin{itemize}
\item Easier for collaborators to clone and build
\item No submodule learning curve for team members
\item Single repository simplifies CI/CD
\item Good balance of independence and integration
\end{itemize}

\section{Implementation Steps}

\subsection{Immediate Next Steps}

Based on your current situation, here's what I recommend:

\begin{enumerate}
\item \textbf{Create GitHub Repository}: Create a new repository for vm\_freertos
\item \textbf{Choose Strategy}: Pick git submodule for research independence
\item \textbf{Set Up Submodule}: Follow the submodule setup process above
\item \textbf{Test Build}: Verify that CAmkES still builds correctly
\item \textbf{Document Workflow}: Create README with build and development instructions
\end{enumerate}

\subsection{Repository Structure}

Your vm\_freertos repository should have this structure:

\begin{lstlisting}[caption=Recommended Repository Structure]
vm_freertos/
├── README.md                    # Project documentation
├── CMakeLists.txt              # Build configuration  
├── vm_minimal.camkes           # CAmkES system definition
├── settings.cmake              # Build settings
├── qemu-arm-virt/             # Platform-specific files
│   ├── devices.camkes         # Device configuration
│   └── freertos_build/        # FreeRTOS source
│       ├── minimal_main_virt.c
│       ├── minimal_startup_virt.S
│       ├── minimal_virt.ld
│       └── minimal_virt.mk
├── docs/                      # Research documentation
│   ├── memory-layout.md       # Memory architecture notes
│   ├── debugging-guide.md     # Common issues and solutions
│   └── research-notes.md      # Your research observations
└── .gitignore                 # Git ignore patterns
\end{lstlisting}

\section{Advanced Git Workflows}

\subsection{Branching Strategy for Research}

\begin{lstlisting}[caption=Research Branching Workflow]
# Main development
git checkout main

# Create experiment branches
git checkout -b experiment/memory-debugging
git checkout -b experiment/performance-optimization
git checkout -b experiment/security-features

# Create paper-specific branches
git checkout -b paper/vm-performance-analysis
git checkout -b paper/security-evaluation

# Tag important milestones
git tag v1.0-initial-implementation
git tag v1.1-paper-submission
git tag v2.0-improved-performance
\end{lstlisting}

\subsection{Collaboration Workflow}

\begin{lstlisting}[caption=Collaboration with Advisors/Peers]
# Share specific experiments
git checkout experiment/memory-debugging
git push origin experiment/memory-debugging

# Create pull requests for review
# Advisors can review changes before merging

# Maintain clean main branch
git checkout main
git merge --no-ff experiment/memory-debugging
git tag v1.2-memory-debug-complete
\end{lstlisting}

\section{Documentation Strategy}

\subsection{README Template}

\begin{lstlisting}[caption=README.md Template]
# FreeRTOS VM for seL4/CAmkES

## Overview
Custom FreeRTOS virtualization implementation for seL4 microkernel research.

## Quick Start
```bash
# From camkes-vm-examples directory
mkdir build && cd build
source ../../sel4-dev-env/bin/activate
../init-build.sh -DCAMKES_VM_APP=vm_freertos -DPLATFORM=qemu-arm-virt -DSIMULATION=1 -DAARCH64=1
ninja
./simulate
```

## Research Focus
- Real-time OS virtualization
- Memory isolation analysis  
- Performance characterization
- Security evaluation

## Publications
- [Paper 1]: "Performance Analysis of FreeRTOS on seL4"
- [Paper 2]: "Security Evaluation of Microkernel Virtualization"

## Citation
If you use this code in your research, please cite:
```
@inproceedings{your2024freertos,
  title={Your Paper Title},
  author={Your Name},
  booktitle={Conference Name},
  year={2024}
}
```
\end{lstlisting}

\section{Backup and Recovery Strategy}

\subsection{Multiple Remote Strategy}

\begin{lstlisting}[caption=Multiple Backup Remotes]
# Add multiple remotes for redundancy
git remote add github git@github.com:username/vm_freertos.git
git remote add gitlab git@gitlab.com:username/vm_freertos.git
git remote add university git@git.university.edu:username/vm_freertos.git

# Push to all remotes
git push github main
git push gitlab main  
git push university main

# Or create alias for pushing to all
git config alias.pushall '!git push github main && git push gitlab main && git push university main'
git pushall
\end{lstlisting}

\subsection{Research Archive Strategy}

\begin{lstlisting}[caption=Creating Research Archives]
# Create archives for paper submissions
git archive --format=tar.gz --prefix=vm_freertos-v1.0/ v1.0 > vm_freertos-v1.0.tar.gz

# Create clean code for sharing
git checkout paper/performance-analysis
git archive --format=zip --prefix=freertos-vm-code/ HEAD > paper-code.zip
\end{lstlisting}

\section{Integration with CAmkES Workflow}

\subsection{Build System Integration}

The submodule approach integrates seamlessly with the existing CAmkES build system:

\begin{lstlisting}[caption=Build System Compatibility]
# No changes needed to existing build commands
cd camkes-vm-examples/build
source ../../sel4-dev-env/bin/activate
../init-build.sh -DCAMKES_VM_APP=vm_freertos -DPLATFORM=qemu-arm-virt -DSIMULATION=1 -DAARCH64=1
ninja
./simulate

# Git operations work independently
cd ../projects/vm-examples/apps/Arm/vm_freertos
git status
git commit -m "Latest research changes"
git push
\end{lstlisting}

\section{Conclusion}

Setting up vm\_freertos as a git-managed subproject provides significant benefits for research code management, collaboration, and publication. The git submodule approach offers the best balance of independence and integration for academic research workflows.

Key benefits include:
\begin{itemize}
\item \textbf{Professional Code Management}: Industry-standard practices for research code
\item \textbf{Collaboration Ready}: Easy sharing with advisors, peers, and reviewers
\item \textbf{Publication Friendly}: Clean, standalone code for paper submissions
\item \textbf{Experimental Flexibility}: Branching for different research directions
\item \textbf{Backup Security}: Multiple remote repositories for safety
\end{itemize}

The investment in proper git setup will pay dividends throughout your PhD research and beyond.

\end{document}